\def\C{C}
\def\Ccomp{C^{\rm Comp}}
\def\Ccomm{C^{\rm Comm}}
\def\Csqm{C_{\rm SQM}}
\def\Csvrg{C_{\rm SVRG}}
\def\Ccompsqm{C^{\rm Comp}_{\rm SQM}}
\def\Ccommsqm{C^{\rm Comm}_{\rm SQM}}
\def\Ccompsvrg{C^{\rm Comp}_{\rm SVRG}}
\def\Ccommsvrg{C^{\rm Comm}_{\rm SVRG}}
\def\Tout{T^{\rm outer}}
\def\Tin{T^{\rm inner}}
\def\Toutsqm{T^{\rm outer}_{\rm SQM}}
\def\Toutsvrg{T^{\rm outer}_{\rm SVRG}}

\section{Appendix B: Complexity analysis}
\label{proofs-complexity}
Let us use the notations of section~\ref{distr} given around (\ref{comm}). We define the overall cost of any distributed algorithm as
\begin{eqnarray}
[(c_1\frac{nz}{p} + c_2m)\Tin + c_3 \gamma m \log_2 P]\Tout,
\label{costanal}
\end{eqnarray}
where $\Tout$ is the number of outer iterations, $\Tin$ is the number of inner iterations at each node before communication happens and  $c_1$ and $c_2$ denote the number of passes over the data and $m$-dimensional dot products per inner iteration respectively. For communication, we assume an AllReduce binary tree as described in Agarwal et al~\yrcite{agarwal2011} without pipelining. As a result, we get a multiplicative factor of $log_2 P$ in our cost. $\gamma$ is the ratio of computation to communication speed. For sparse datasets $\gamma$ is very large. $c_3$ is the number of $m$-dimensional vectors (gradients, Hessian-vector computations etc.) we need to communicate.

\begin{table}[ht]
\caption{Value of cost parameters} % title of Table
\centering % used for centering table
\begin{tabular}{c c c c c c c} % centered columns (4 columns)
\hline\hline %inserts double horizontal lines
Method & $c_1$ & $c_2$ & $c_3$ & $\Tin$ \\ %[0.5ex] % inserts table
%heading
\hline
SQM & $2$ & $\approx 5-10$ & $1$ & $1$ \\
Our & $1.2$ & $0.2$ & $2$ & $\hat{k}$ \\
\hline
\end{tabular}
\label{tab:sqmparams}
\end{table}

The values of different parameters for $SQM$ implemented using TRON and our approach implemented using SVRG are given in Table~\ref{tab:sqmparams}. $\Toutsqm$ is the number of overall conjugate gradient iterations plus gradient computations.

Since dense dot products are extremely fast and $c_3$ is a small number for both the approaches, we ignore it for simplicity. Now for our method to have lesser cost than $SQM$, we can use (\ref{costanal}) to get the condition,
%we need $\Csvrg \leq \Csqm$. Plugging in the parameter values and rearranging,
\begin{equation}
(1.2\hat{k}\Toutsvrg - 2\Toutsqm)\frac{nz}{P}  \leq  (\Toutsqm - 2 \Toutsvrg) \gamma m \log_2 P
\end{equation}
Ignoring $\Toutsqm$ on the left side of this inequality and rearranging, we get the looser condition,
\begin{equation}
\frac{nz}{m}  \leq  \frac{\gamma P\log_2 P}{\hat{k}} \frac{1}{1.2}(\frac{\Toutsqm}{\Toutsvrg} - 2 )
\end{equation}
Assuming $\Toutsqm > 3.2\Toutsvrg$, we arrive at the final condition in equation~(\ref{comm}). 
%Note that in our experiments (section~\ref{distr}), the savings we get in the outer iterations is typically more than the factor $3.2$.
